% -*- latex -*-
%%%%%%%%%%%%%%%%%%%%%%%%%%%%%%%%%%%%%%%%%%%%%%%%%%%%%%%%%%%%%%%%
%%%%%%%%%%%%%%%%%%%%%%%%%%%%%%%%%%%%%%%%%%%%%%%%%%%%%%%%%%%%%%%%
%%%%
%%%% This text file is part of the lecture slides for
%%%% `Parallel Programming in MPI and OpenMP'
%%%% by Victor Eijkhout, copyright 2012-2021
%%%%
%%%% coursemacs.tex : macros for the lecture slides
%%%% THIS NEEDS TO BE MERGED INTO slidemacs AS MUCH AS POSSIBLE
%%%%
%%%%%%%%%%%%%%%%%%%%%%%%%%%%%%%%%%%%%%%%%%%%%%%%%%%%%%%%%%%%%%%%
%%%%%%%%%%%%%%%%%%%%%%%%%%%%%%%%%%%%%%%%%%%%%%%%%%%%%%%%%%%%%%%%

\usepackage{pslatex}
\usepackage{verbatim,wrapfig}

\usepackage{amsmath}
\usepackage{array,booktabs,multirow,multicol}
\newcolumntype{R}{>{\hbox to 1.2em\bgroup\hss}{r}<{\egroup}}
\newcolumntype{T}{>{\hbox to 8em\bgroup}{c}<{\hss\egroup}}

\def\verbatimsnippet#1{\begingroup\small \verbatiminput{snippets/#1}\endgroup}
\usepackage{hyperref}

\newdimen\unitindent \unitindent=20pt
\usepackage[algo2e,noline,noend]{algorithm2e}
\newenvironment{displayalgorithm}
 {\par
  \begin{algorithm2e}[H]\leftskip=\unitindent \parskip=0pt\relax
  \DontPrintSemicolon
  \SetKwInOut{Input}{Input}\SetKwInOut{Output}{Output}
 }
 {\end{algorithm2e}\par}
\newenvironment{displayprocedure}[2]
 {\everymath{\strut}
  \begin{procedure}[H]\leftskip=\unitindent\caption{#1(#2)}}
 {\end{procedure}}
\def\sublocal{_{\mathrm\scriptstyle local}}

%%%%
%%%% Comment environment
%%%%
\usepackage{comment}
%\input longshort
\specialcomment{exercise}{
  \def\CommentCutFile{exercise.cut}
  \def\PrepareCutFile{%
        \immediate\write\CommentStream{\noexpand\begin{frame}{Exercise}}}
  \def\FinalizeCutFile{%
        \immediate\write\CommentStream{\noexpand\end{frame}}}
  }{}
\excludecomment{pcse}
\excludecomment{book}

%%%%
%%%% Outlining
%%%%
\usepackage{outliner}
\OutlineLevelStart 0{\part{#1}\frame{\partpage}}    %{\frame{\part{#1}\Huge\bf #1}}
%\OutlineLevelStart 1{\section{#1}}
\OutlineLevelStart 1{\frame{\section{#1}\Large\bf#1}}
\def\sectionframe#{\Level 1 }
\setcounter{tocdepth}{2}
\usepackage{framed}
\colorlet{shadecolor}{blue!15}
\OutlineLevelStart 2{\subsection{#1}
  \frame{\begin{shaded}\large #1\end{shaded}}}
%% \OutlineLevelCont 2{\end{frame}\begin{frame}{#1}}
%% \OutlineLevelEnd 2{\end{frame}}
%\OutlineTracetrue

\newcommand\coursepart[1]{
  %\addcontentsline{toc}{part}{Section: #1}
  \begin{frame}{}
    \setbox0=\hbox to \hsize{\hfil\Huge{#1}\hfil}
    \dimen0=\vsize
    \advance\dimen0 by -\ht0 \advance\dimen0 by -\dp0
    \divide\dimen0 by 2
    \hbox{}
    \vskip \dimen0
    \box0
    \vskip \dimen0
    \hbox{}
  \end{frame}
}

\def\underscore{_}
\def\mpiRoutineRef#1{\RoutineRefStyle\verbatiminput{#1}}
\def\RoutineRefStyle{\scriptsize}
\def\protoslide#{\bgroup \catcode`\_=12
  \afterassignment\protoslideinclude \def\protoname}
\def\protoslideinclude{%
  \begin{frame}[containsverbatim]\frametitle{\texttt\protoname}
    \scriptsize
    \edef\tmp{\lowercase{\def\noexpand\standardroutine{\protoname}}}\tmp
    \IfFileExists
        {standard/\standardroutine.tex}
        {\input{standard/\standardroutine}}
        {% if no standard file, then maybe handwritten file
          \IfFileExists
              {mpireference/\protoname.tex}
              {\verbatiminput{mpireference/\protoname}}{}}
        %\expandafter\mpiRoutineRef\expandafter{\protoname}
  \end{frame}\egroup
}

\def\innocentcharacters{% move this to commen and unify
  \catcode`\_=12 \catcode`\#=12
  \catcode`\>=12 \catcode`\<=12
  \catcode`\&=12 \catcode`\^=12 \catcode`\~=12
}
\newcommand\slackpoll[1]{\begingroup \par \tiny\texttt{/poll #1} \par \endgroup}
\def\slackpollTF#1{\begingroup \def\terminator{#1}
  \innocentcharacters \catcode`\{=12 \catcode`\}=12
  \edef\tmp{\def\noexpand\slackpollTFp####1\terminator
    {\noexpand\par
      \noexpand\tiny \noexpand\texttt{/poll "####1" "T" "F"}
      \noexpand\par \endgroup}}\tmp
  \slackpollTFp}

\newcounter{excounter}
\newcounter{revcounter}
\newenvironment
    {reviewframe}
    {\begin{frame}[containsverbatim]
        \frametitle{Review \arabic{revcounter}}
        \refstepcounter{revcounter}}
    {\end{frame}}
\newenvironment
    {exerciseframe}[1][]
    {\def\optfile{#1}
      \ifx\optfile\empty\else\message{VLE EXERCISE #1}%\index[programming]{#1}
      \fi
      \begin{frame}[containsverbatim]
        \ifx\optfile\empty
        \frametitle{Exercise \arabic{excounter}}
        \else
        \frametitle{Exercise \arabic{excounter} (\tt{#1})}
        \fi
        \refstepcounter{excounter}}
    {\end{frame}}
\newenvironment
    {optexerciseframe}[1][]
    {\begin{frame}[containsverbatim]
        \def\optfile{#1}
        \ifx\optfile\empty
        \frametitle{Exercise (optional) \arabic{excounter}}
        \else
        \frametitle{Exercise (optional) \arabic{excounter} (\tt{#1})}
        \fi
        \refstepcounter{excounter}}
    {\end{frame}}

%\advance\textwidth by 1in
%\advance\oddsidemargin by -.5in

%%%%
%%%% Beamer customization
%%%%

\setbeamertemplate{title page}{
  \begin{picture}(0,0)
    \put(-10,10){\includegraphics[scale=.45]{tacctitle}}
    \put(50,-50){
      \begin{minipage}{10cm}
        \usebeamerfont{title}{\inserttitle\par\insertauthor\par\insertdate\par}
      \end{minipage}
    }
  \end{picture}
}

%%%%
%%%% Indexing, mostly disabled
%%%%

% the 
% \usepackage{imakeidx}
% line is in the master file
%
%% \makeindex
%% \makeindex[name=programming,title=Programming exercises]

\let\indexterm\emph
\let\indextermtt\n
\let\indextermfunction\indextermtt
\newcommand{\indextermsubh}[2]{\emph{#1-#2}}

\iffalse
\newcommand{\indextermp}[1]{\emph{#1s}}
\newcommand{\indextermsub}[2]{\emph{#1 #2}}
\newcommand{\indextermsubp}[2]{\emph{#1 #2s}}
\newcommand{\indextermbus}[2]{\emph{#1 #2}}
\newcommand{\indextermstart}[1]{\emph{#1}}
\newcommand{\indextermend}[1]{}
\newcommand{\indexstart}[1]{}
\newcommand{\indexend}[1]{}
\makeatletter
\newcommand\indexac[1]{\emph{\ac{#1}}}
\newcommand\indexacp[1]{\emph{\ac{#1}}}
\newcommand\indexacf[1]{\emph{\acf{#1}}}
\newcommand\indexacstart[1]{}
\newcommand\indexacend[1]{}
\makeatother

\let\indexmpishow\n
\let\indexmpidef\n
\let\indexmpiex\n
\let\indexmpi\n
\let\indexompshow\n
\let\indexompdef\n
\global\let\indexompshowdef\indexompdef
\let\indexompex\n
\let\indexomp\n
\let\mpitoindex\n
\let\mpitoindexbf\n
\let\mpitoindexit\n
\let\omptoindex\n
\let\omptoindexbf\n
\let\omptoindexit\n
%\let\indextermlet#1{\emph{#1}\index{#1|textbf}}
\let\indextermtt\n
%\let\indextermttlet\indexmpilet
\let\indexcommand\n
\let\indexclause\n
\let\indexclauselet\n
\let\indexclauseoption\n

\let\indexmpiref\indexmpishow
\let\indexmpidef\indexmpishow
\let\indexmpldef\indexmpishow
\let\indexmplref\indexmpishow
\let\indexmplshow\indexmpishow
\let\indexompshow\indexmpishow
\let\indexpetscshow\indexmpishow
\fi


%%%%
%%%% Environments
%%%%
\newcounter{slidecount}
\setcounter{slidecount}{1}
\newenvironment
    {numberedframe}[1]
    {\begin{frame}[containsverbatim]{\arabic{slidecount}.\ #1}
        \refstepcounter{slidecount}
    }
    {\end{frame}}
\newenvironment{question}{\begin{quotation}\textbf{Question.\ }}{\end{quotation}}
\newenvironment{fortrannote}
  {\begin{quotation}\noindent\textsl{Fortran note.\kern1em}\ignorespaces}
  {\end{quotation}}
\newenvironment{pythonnote}
  {\begin{quotation}\noindent\textsl{Python note.\kern1em}\ignorespaces}
  {\end{quotation}}
\newenvironment{taccnote}
  {\begin{quotation}\noindent\textsl{TACC note.\kern1em}\ignorespaces}
  {\end{quotation}}
\newenvironment{mpifour}
  {\begin{quotation}\textbf{MPI-4:\ }\ignorespaces}{\end{quotation}}

